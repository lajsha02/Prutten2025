\documentclass{article}\usepackage{noweb}\pagestyle{noweb}\noweboptions{}\begin{document}\nwfilename{Suitcase.nw}\nwbegindocs{0}\title{Resväska som \emph{Composite} i Java}% ===> this file was generated automatically by noweave --- better not edit it
\author{Lisa Dahl och Mostafa Shihadeh}

\maketitle
\tableofcontents

\section{Introduktion}
Vi vill implementera uppgift 1 (resväska enligt \emph{Composite}-mönstret)
med litterär programmering.
Vi bygger tre klasser: \texttt{Component} (abstrakt), \texttt{Leaf} (pryl) och
\texttt{Composite} (behållare), samt ett testprogram \texttt{Client}.

\subsection{Designval}
\begin{itemize}
  \item \texttt{Component} bär gemensamma attribut: \texttt{name} och \texttt{weight} (egen vikt).
  \item \texttt{Leaf} representerar en enskild pryl. \texttt{getWeight()} returnerar bara dess egen vikt.
  \item \texttt{Composite} representerar en behållare med barn. \texttt{getWeight()} summerar
        behållarens egen vikt och alla barns vikter. \texttt{toString()} traverserar rekursivt.
  \item Metoderna \texttt{add} och \texttt{remove} finns bara i \texttt{Composite} (inte i \texttt{Component}).
\end{itemize}

\subsection{Bygga och kompilera}
Vi vill skriva en byggfil för GNU Make. Själva, maskin-genererade reglerna
läggs i \texttt{Suitcase.mk} som tanglas ur denna \texttt{.nw}-fil och sedan
\emph{inkluderas} från toppens \texttt{Makefile}.

Först lägger vi grundmålen och PDF-reglerna:

\nwenddocs{}\nwbegincode{1}\moddef{Suitcase.mk}\endmoddef\nwstartdeflinemarkup\nwenddeflinemarkup
TARGETS= Suitcase.pdf Suitcase.mk
all: classes Suitcase.pdf

Suitcase.pdf:  Suitcase.tex
        pdflatex -interaction=nonstopmode -halt-on-error Suitcase.tex
        pdflatex -interaction=nonstopmode -halt-on-error Suitcase.tex

Suitcase.tex: Suitcase.nw
        noweave -latex Suitcase.nw > Suitcase.tex
\nwendcode{}\nwbegindocs{2}\nwdocspar

Därefter tanglar vi ut varje \texttt{.java}-fil ur den litterära källan
(samt bäddar in radmarkörer så att \texttt{noerr.pl} kan mappa fel till \texttt{.nw}):

\nwenddocs{}\nwbegincode{3}\moddef{Suitcase.mk}\plusendmoddef\nwstartdeflinemarkup\nwenddeflinemarkup
Component.java: Suitcase.nw
        notangle -L'//line %L "%F"%N' -RComponent.java Suitcase.nw > Component.java

Leaf.java: Suitcase.nw
        notangle -L'//line %L "%F"%N' -RLeaf.java Suitcase.nw > Leaf.java

Composite.java: Suitcase.nw
        notangle -L'//line %L "%F"%N' -RComposite.java Suitcase.nw > Composite.java

Client.java: Suitcase.nw
        notangle -L'//line %L "%F"%N' -RClient.java Suitcase.nw > Client.java
\nwendcode{}\nwbegindocs{4}\nwdocspar

Nu lägger vi sektionen för att kompilera och köra Java:

\nwenddocs{}\nwbegincode{5}\moddef{Suitcase.mk}\plusendmoddef\nwstartdeflinemarkup\nwenddeflinemarkup
.PHONY: classes run clean-Suitcase
classes: Component.java Leaf.java Composite.java Client.java
        @if [ -x ./noerr.pl ]; then ./noerr.pl javac *.java; else javac *.java; fi

run: classes
        java Client
\nwendcode{}\nwbegindocs{6}\nwdocspar

Slutligen städreglerna:

\nwenddocs{}\nwbegincode{7}\moddef{Suitcase.mk}\plusendmoddef\nwstartdeflinemarkup\nwenddeflinemarkup
clean: clean-Suitcase
clean-Suitcase:
        rm -f Suitcase.tex Suitcase.aux Suitcase.log Suitcase.toc
        rm -f Suitcase.mk Suitcase.pdf
        rm -f *.class *.java
\nwendcode{}\nwbegindocs{8}\nwdocspar

\section{Kod}
I det här avsnittet definierar vi de fyra Javafilerna som tänglas ut från denna \texttt{.nw}-fil.

\subsection{\texttt{Klassen Component}}
\noindent \textbf{Ansvar:} bas-klass med namn och egenvikt samt abstrakta metoder.

Filen {\Tt{}Component.java\nwendquote} innehåller definitionen av klassen \texttt{Component}.
Översiktligt ser den ut så här:
\nwenddocs{}\nwbegincode{9}\moddef{Component.java}\endmoddef\nwstartdeflinemarkup\nwenddeflinemarkup
public abstract class Component \{
  \LA{}Component attributes\RA{}
  \LA{}Component constructor\RA{}
  \LA{}Component methods\RA{}
\}
\nwendcode{}\nwbegindocs{10}\nwdocspar

\section{Konstruktor}
Vi skapar en konstruktor för klassen \texttt{Konstruktor} som initierar alla 
attribut. Vi vill kunna skapa en behållare med ett namn och en vikt.

Vi vill att klassen \texttt{Component} ska ha följande attribut:
\begin{itemize}
  \item \texttt{name}: namn på komponenten (sträng)
  \item \texttt{weight}: egen vikt i kg (flyttal)
\end{itemize}

Dessa attribut ska vara skyddade och slutgiltiga (protected final) för att förhindra
ändring efter konstruktion.

\nwenddocs{}\nwbegincode{11}\moddef{Component attributes}\endmoddef\nwstartdeflinemarkup\nwenddeflinemarkup
  protected final String name;
  protected final double weight; 
\nwendcode{}\nwbegindocs{12}\nwdocspar
Vi lägger även till kontroller i konstruktorn för att säkerställa att
namnet inte är null eller tomt, och att vikten är icke-negativ.

\nwenddocs{}\nwbegincode{13}\moddef{Component constructor}\endmoddef\nwstartdeflinemarkup\nwenddeflinemarkup
protected Component(String name, double weight) \{
    if (name == null || name.isBlank()) throw new IllegalArgumentException("name");
    if (weight < 0) throw new IllegalArgumentException("weight");
    this.name = name;
    this.weight = weight;
  \}
\nwendcode{}\nwbegindocs{14}\nwdocspar
\section{Metoder}
Vi definierar metoderna \texttt{getName}, \texttt{getOwnWeight} i klassen \texttt{Component}, samt de abstrakta metoderna \texttt{getWeight} och \texttt{toString}.:
\begin{itemize}
  \item \texttt{getName}: returnerar komponentens namn.
  \item \texttt{getOwnWeight}: returnerar komponentens egen vikt (utan barn).
  \item \texttt{getWeight}: abstrakt metod som ska returnera totalvikten (inklusive barn).
  \item \texttt{toString}: abstrakt metod som ska returnera en strängrepresentation av komponenten.
\end{itemize}

\nwenddocs{}\nwbegincode{15}\moddef{Component methods}\endmoddef\nwstartdeflinemarkup\nwenddeflinemarkup
  public String getName() \{ return name; \}
  public double getOwnWeight() \{ return weight; \}
  public abstract double getWeight();
  public abstract String toString();
\nwendcode{}\nwbegindocs{16}\nwdocspar

\subsection{\texttt{Leaf}}
\noindent \textbf{Ansvar:} en enskild pryl att packa.

Filen {\Tt{}Leaf.java\nwendquote} innehåller definitionen av klassen \texttt{Leaf}.
Översiktligt ser den ut så här:

\nwenddocs{}\nwbegincode{17}\moddef{Leaf.java}\endmoddef\nwstartdeflinemarkup\nwenddeflinemarkup
public class Leaf extends Component \{
  \LA{}Leaf constructor\RA{}
  \LA{}Leaf methods\RA{}
\}
\nwendcode{}\nwbegindocs{18}\nwdocspar

\subsubsection{Konstruktor}
Konstruktorn tar namn och egen vikt och skickar vidare till basklassen \texttt{Component}.

\nwenddocs{}\nwbegincode{19}\moddef{Leaf constructor}\endmoddef\nwstartdeflinemarkup\nwenddeflinemarkup
  public Leaf(String name, double weight) \{
    super(name, weight);
  \}
\nwendcode{}\nwbegindocs{20}\nwdocspar

\subsubsection{Metoder}
Totalvikten för ett löv är samma som dess egen vikt. \texttt{toString} beskriver prylen.

\nwenddocs{}\nwbegincode{21}\moddef{Leaf methods}\endmoddef\nwstartdeflinemarkup\nwenddeflinemarkup
  @Override
  public double getWeight() \{
    return this.weight;
  \}

  @Override
  public String toString() \{
    return this.name + " (" + this.weight + " kg)";
  \}
\nwendcode{}\nwbegindocs{22}\nwdocspar

\subsection{\texttt{Composite}}
\noindent \textbf{Ansvar:} en behållare som kan innehålla andra \texttt{Component}.
Egen vikt \emph{plus} alla barns vikter utgör totalvikten.

Filen {\Tt{}Composite.java\nwendquote} innehåller definitionen av klassen \texttt{Composite}.
Översiktligt ser den ut så här:

\nwenddocs{}\nwbegincode{23}\moddef{Composite.java}\endmoddef\nwstartdeflinemarkup\nwenddeflinemarkup
import java.util.ArrayList;
import java.util.Collections;
import java.util.List;

public class Composite extends Component \{
  \LA{}Composite fields\RA{}
  \LA{}Composite constructor\RA{}
  \LA{}Composite child-management\RA{}
  \LA{}Composite weight\RA{}
  \LA{}Composite toString\RA{}
\}
\nwendcode{}\nwbegindocs{24}\nwdocspar

\subsubsection{Fält}
Vi lagrar barn i en muterbar lista, men exponerar en oföränderlig vy utåt.

\nwenddocs{}\nwbegincode{25}\moddef{Composite fields}\endmoddef\nwstartdeflinemarkup\nwenddeflinemarkup
  private final List<Component> children = new ArrayList<>();
\nwendcode{}\nwbegindocs{26}\nwdocspar

\subsubsection{Konstruktor}
Konstruktorn tar behållarens namn och egen vikt.

\nwenddocs{}\nwbegincode{27}\moddef{Composite constructor}\endmoddef\nwstartdeflinemarkup\nwenddeflinemarkup
  public Composite(String name, double ownWeight) \{
    super(name, ownWeight);
  \}
\nwendcode{}\nwbegindocs{28}\nwdocspar

\subsubsection{Barnhantering (\texttt{add}/\texttt{remove}/\texttt{getChildren})}
Vi kan lägga till och ta bort barn, och ge tillbaka en oföränderlig vy av listan.

\nwenddocs{}\nwbegincode{29}\moddef{Composite child-management}\endmoddef\nwstartdeflinemarkup\nwenddeflinemarkup
  public void add(Component c) \{
    if (c == null) throw new IllegalArgumentException("child");
    children.add(c);
  \}

  public void remove(Component c) \{
    children.remove(c);
  \}

  public List<Component> getChildren() \{
    return Collections.unmodifiableList(children);
  \}
\nwendcode{}\nwbegindocs{30}\nwdocspar

\subsubsection{Totalvikt (\texttt{getWeight})}
Totalvikt = egen vikt + summan av alla barns totalvikter (rekursivt).

\nwenddocs{}\nwbegincode{31}\moddef{Composite weight}\endmoddef\nwstartdeflinemarkup\nwenddeflinemarkup
  @Override
  public double getWeight() \{
    double sum = this.weight;
    for (Component c : children) \{
      sum += c.getWeight();
    \}
    return sum;
  \}
\nwendcode{}\nwbegindocs{32}\nwdocspar

\subsubsection{\texttt{toString}}
Vi bygger en rekursiv beskrivning där barn listas inom hakparenteser.

\nwenddocs{}\nwbegincode{33}\moddef{Composite toString}\endmoddef\nwstartdeflinemarkup\nwenddeflinemarkup
  @Override
  public String toString() \{
    StringBuilder sb = new StringBuilder();
    sb.append(this.name).append(" (").append(this.weight).append(" kg) [");
    for (int i = 0; i < children.size(); i++) \{
      sb.append(children.get(i).toString());
      if (i < children.size() - 1) sb.append(", ");
    \}
    sb.append("]");
    return sb.toString();
  \}
\nwendcode{}\nwbegindocs{34}\nwdocspar
\subsection{\texttt{Client}}
\noindent \textbf{Ansvar:} bygga en resväska med minst tre nivåer och minst tio prylar,
skriva ut totalvikt och innehåll, ta bort några objekt och skriva ut igen.

Filen {\Tt{}Client.java\nwendquote} innehåller testprogrammet. Översiktligt ser det ut så här:

\nwenddocs{}\nwbegincode{35}\moddef{Client.java}\endmoddef\nwstartdeflinemarkup\nwenddeflinemarkup
public class Client \{
  public static void main(String[] args) \{
    \LA{}Client build-structure\RA{}
    \LA{}Client print-before\RA{}
    \LA{}Client removals\RA{}
    \LA{}Client print-after\RA{}
  \}
\}
\nwendcode{}\nwbegindocs{36}\nwdocspar

\subsubsection{Bygg upp strukturen (\texttt{suitcase})}
Vi skapar roten (resväskan), underbehållare och löv.

\nwenddocs{}\nwbegincode{37}\moddef{Client build-structure}\endmoddef\nwstartdeflinemarkup\nwenddeflinemarkup
    // Roten: själva resväskan (egen vikt 2.3 kg)
    Composite suitcase = new Composite("Resväska", 2.3);

    // Större plagg (löven)
    Leaf tshirt1 = new Leaf("T-shirt vit", 0.18);
    Leaf tshirt2 = new Leaf("T-shirt svart", 0.19);
    Leaf jeans   = new Leaf("Jeans", 0.75);
    Leaf chinos  = new Leaf("Chinos", 0.55);
    Leaf bok     = new Leaf("Pocketbok", 0.28);

    // Necessär (behållare) med egen vikt 0.12 kg och innehåll
    Composite necessar = new Composite("Necessär", 0.12);
    Leaf tvål     = new Leaf("Tvål", 0.09);
    Leaf schampo  = new Leaf("Schampo", 0.22);
    Leaf borste   = new Leaf("Tandborste", 0.03);
    Leaf tandkräm = new Leaf("Tandkräm", 0.11);
    necessar.add(tvål);
    necessar.add(schampo);
    necessar.add(borste);
    necessar.add(tandkräm);

    // Påse i necessären (tredje nivån)
    Composite påse = new Composite("Påse", 0.01);
    Leaf hårspännen = new Leaf("Hårspännen (10 st)", 0.02);
    påse.add(hårspännen);
    necessar.add(påse);

    // Mindre väska för elektronik (behållare) — nivå två
    Composite techbag = new Composite("Tech-väska", 0.20);
    Leaf laddare  = new Leaf("Laddare", 0.15);
    Leaf hörlurar = new Leaf("Hörlurar", 0.08);
    Leaf powerbank = new Leaf("Powerbank", 0.18);
    techbag.add(laddare);
    techbag.add(hörlurar);
    techbag.add(powerbank);

    // Packa allt i resväskan
    suitcase.add(tshirt1);
    suitcase.add(tshirt2);
    suitcase.add(jeans);
    suitcase.add(chinos);
    suitcase.add(bok);
    suitcase.add(necessar);
    suitcase.add(techbag);
\nwendcode{}\nwbegindocs{38}\nwdocspar

\subsubsection{Skriv ut totalvikt och innehåll (före borttagning)}
\nwenddocs{}\nwbegincode{39}\moddef{Client print-before}\endmoddef\nwstartdeflinemarkup\nwenddeflinemarkup
    System.out.printf("Totalvikt före borttagning: %.2f kg%n", suitcase.getWeight());
    System.out.println("Innehåll före borttagning:");
    System.out.println(suitcase.toString());
\nwendcode{}\nwbegindocs{40}\nwdocspar

\subsubsection{Ta bort några saker/behållare}
Vi tar bort en pryl i en behållare, hela teknikväskan och påsen i necessären.

\nwenddocs{}\nwbegincode{41}\moddef{Client removals}\endmoddef\nwstartdeflinemarkup\nwenddeflinemarkup
    techbag.remove(powerbank);      // ta bort en pryl i behållare
    suitcase.remove(techbag);       // ta bort hela behållaren
    necessar.remove(påse);          // ta bort tredje nivåns behållare
\nwendcode{}\nwbegindocs{42}\nwdocspar

\subsubsection{Skriv ut totalvikt och innehåll (efter borttagning)}
\nwenddocs{}\nwbegincode{43}\moddef{Client print-after}\endmoddef\nwstartdeflinemarkup\nwenddeflinemarkup
    System.out.printf("Totalvikt efter borttagning: %.2f kg%n", suitcase.getWeight());
    System.out.println("Innehåll efter borttagning:");
    System.out.println(suitcase.toString());
\nwendcode{}\nwbegindocs{44}\nwdocspar


\section{Kommentarer om korrekthet och mönstertrohet}
\begin{itemize}
  \item \texttt{getWeight()} anropat på ett \texttt{Composite}-objekt itererar rekursivt över alla barn
        (och deras barn), vilket följer \emph{Composite}-mönstrets idé.
  \item \texttt{toString()} på \texttt{Composite} bygger en rekursiv presentation av hela trädet.
  \item \texttt{Leaf} saknar \texttt{add}/\texttt{remove} och är därmed ett \emph{löv}.
  \item Egen vikt för behållare ingår alltid i totalvikten, som efterfrågats i labbtexten.
\end{itemize}

\section{Bygginstruktion}
Kör \texttt{make} för att generera \texttt{Suitcase.pdf} och \texttt{Suitcase.mk}.
Kör \texttt{make classes} för att kompilera Javafilerna och \texttt{make run}
för att köra demon av \texttt{Client}.
\nwenddocs{}\end{document}

