\documentclass{article}\usepackage{noweb}\pagestyle{noweb}\noweboptions{}\begin{document}\nwfilename{HumanFactory.nw}\nwbegindocs{0}\title{Human factory i Java (paket \texttt{human})}% ===> this file was generated automatically by noweave --- better not edit it
\author{Lisa Dahl och Mostafa Shihadeh}

\maketitle
\tableofcontents

\section{Introduktion}
I denna del implementerar vi \emph{Human factory}. Vi skriver en abstrakt
\texttt{Human} och tre konkreta subklasser: \texttt{NonBinary}, \texttt{Woman}
och \texttt{Man}. En statisk fabriksmetod i \texttt{Human} väljer subklass baserat
på personnumrets näst sista tecken: \texttt{'0'} $\Rightarrow$ \texttt{NonBinary};
udda siffra $\Rightarrow$ \texttt{Man}; jämn (men ej \texttt{'0'}) $\Rightarrow$ \texttt{Woman}.

Vi placerar \texttt{Human}, \texttt{NonBinary}, \texttt{Woman}, \texttt{Man}
i paketet \texttt{human}. Testprogrammet \texttt{TestHuman} ligger på nivån
ovan (default\-paketet) och får endast skapa instanser via fabriken. Det ska
inte gå att kompilera \texttt{new NonBinary(...)} eller \texttt{new Human()\{\}}.

\subsection{Bygga och kompilera}
Precis som i uppgift 1 låter vi denna \texttt{.nw} tängla en maskin\-genererad
\texttt{HumanFactory.mk} som toppens \texttt{Makefile} \emph{inkluderar}.

Först lägger vi grundmålen, inkl.\ PDF:

\nwenddocs{}\nwbegincode{1}\moddef{HumanFactory.mk}\endmoddef\nwstartdeflinemarkup\nwenddeflinemarkup
TARGETS= HumanFactory.pdf HumanFactory.mk
all: classes-human HumanFactory.pdf

HumanFactory.pdf: HumanFactory.tex
        pdflatex -interaction=nonstopmode -halt-on-error HumanFactory.tex
        pdflatex -interaction=nonstopmode -halt-on-error HumanFactory.tex

HumanFactory.tex: HumanFactory.nw
        noweave -latex HumanFactory.nw > HumanFactory.tex
\nwendcode{}\nwbegindocs{2}\nwdocspar

Därefter regler för att tängla ut Java\-källorna (\texttt{human/} skapas vid behov)
med radmarkörer (bra med \texttt{noerr.pl}):

\nwenddocs{}\nwbegincode{3}\moddef{HumanFactory.mk}\plusendmoddef\nwstartdeflinemarkup\nwenddeflinemarkup
human/Human.java: HumanFactory.nw
        mkdir -p human
        notangle -L'//line %L "%F"%N' -Rhuman/Human.java HumanFactory.nw > human/Human.java

human/NonBinary.java: HumanFactory.nw
        mkdir -p human
        notangle -L'//line %L "%F"%N' -Rhuman/NonBinary.java HumanFactory.nw > human/NonBinary.java

human/Woman.java: HumanFactory.nw
        mkdir -p human
        notangle -L'//line %L "%F"%N' -Rhuman/Woman.java HumanFactory.nw > human/Woman.java

human/Man.java: HumanFactory.nw
        mkdir -p human
        notangle -L'//line %L "%F"%N' -Rhuman/Man.java HumanFactory.nw > human/Man.java

TestHuman.java: HumanFactory.nw
        notangle -L'//line %L "%F"%N' -RTestHuman.java HumanFactory.nw > TestHuman.java
\nwendcode{}\nwbegindocs{4}\nwdocspar

Kompilera och kör:

\nwenddocs{}\nwbegincode{5}\moddef{HumanFactory.mk}\plusendmoddef\nwstartdeflinemarkup\nwenddeflinemarkup
.PHONY: classes-human run-human clean-HumanFactory
classes-human: human/Human.java human/NonBinary.java human/Woman.java human/Man.java TestHuman.java
        @if [ -x ./noerr.pl ]; then ./noerr.pl javac human/*.java TestHuman.java; else javac human/*.java TestHuman.java; fi

run-human: classes-human
        java TestHuman
\nwendcode{}\nwbegindocs{6}\nwdocspar

Städregler:

\nwenddocs{}\nwbegincode{7}\moddef{HumanFactory.mk}\plusendmoddef\nwstartdeflinemarkup\nwenddeflinemarkup
clean: clean-HumanFactory
clean-HumanFactory:
        rm -f HumanFactory.tex HumanFactory.aux HumanFactory.log HumanFactory.toc
        rm -f HumanFactory.mk HumanFactory.pdf
        rm -f human/*.class *.class
        rm -f TestHuman.java human/*.java
        rmdir human 2>/dev/null || true
\nwendcode{}\nwbegindocs{8}\nwdocspar

\section{Kod}
Här följer klasserna i paketet \texttt{human} och testprogrammet. Varje fil
presenteras med en översikt och därefter delsteg (chunks) i samma stil som i uppgift 1.

\subsection{\texttt{Human} (abstrakt)}
\noindent \textbf{Ansvar:} bära gemensamma fält (\texttt{name}, \texttt{pnr}) och erbjuda fabriksmetoden.

Filen {\Tt{}human/Human.java\nwendquote} ser översiktligt ut så här:

\nwenddocs{}\nwbegincode{9}\moddef{human/Human.java}\endmoddef\nwstartdeflinemarkup\nwenddeflinemarkup
package human;

public abstract class Human \{
        \LA{}Human fields\RA{}
        \LA{}Human ctor (package-private)\RA{}
        \LA{}Human factory\RA{}
        \LA{}Human methods\RA{}
\}
\nwendcode{}\nwbegindocs{10}\nwdocspar

\subsubsection{Fält}
Vi lagrar namn och personnummer (oföränderliga efter konstruktion).

\nwenddocs{}\nwbegincode{11}\moddef{Human fields}\endmoddef\nwstartdeflinemarkup\nwenddeflinemarkup
        protected final String name;
        protected final String pnr;
\nwendcode{}\nwbegindocs{12}\nwdocspar

\subsubsection{Konstruktor (paket-synlig)}
Konstruktorn är \emph{paket\-synlig} (ingen modifierare) så att kod utanför
\texttt{human} inte kan \texttt{new}a \texttt{Human} eller anonym subklass.

\nwenddocs{}\nwbegincode{13}\moddef{Human ctor (package-private)}\endmoddef\nwstartdeflinemarkup\nwenddeflinemarkup
        Human(String name, String pnr) \{
                if (name == null || name.isBlank()) throw new IllegalArgumentException("name");
                if (pnr == null || pnr.length() < 2) throw new IllegalArgumentException("pnr");
                this.name = name;
                this.pnr = pnr;
        \}
\nwendcode{}\nwbegindocs{14}\nwdocspar

\subsubsection{Fabriksmetod}
Vi följer uppgiftens tumregel på \emph{näst sista tecknet}:
\begin{itemize}
  \item \texttt{'0'} $\Rightarrow$ \texttt{NonBinary}
  \item udda siffra $\Rightarrow$ \texttt{Man}
  \item jämn siffra (ej \texttt{'0'}) $\Rightarrow$ \texttt{Woman}
\end{itemize}
Vi erbjuder den variant som används i exemplen (\texttt{create(name, pnr)}).
Vill man, kan man lägga till en \texttt{create(pnr)} som ger ett standardnamn.

\nwenddocs{}\nwbegincode{15}\moddef{Human factory}\endmoddef\nwstartdeflinemarkup\nwenddeflinemarkup
        public static Human create(String name, String pnr) \{
                if (pnr == null || pnr.length() < 2) throw new IllegalArgumentException("pnr");
                char c = pnr.charAt(pnr.length() - 2); // näst sista tecknet
                if (c == '0') \{
                        return new NonBinary(name, pnr);
                \}
                if (!Character.isDigit(c)) \{
                        throw new IllegalArgumentException("pnr: näst sista tecknet måste vara siffra eller '0'");
                \}
                int d = c - '0';
                if ((d % 2) == 1) \{
                        return new Man(name, pnr);
                \} else \{
                        return new Woman(name, pnr);
                \}
        \}
\nwendcode{}\nwbegindocs{16}\nwdocspar

\subsubsection{Metoder}
Vi exponerar namn och pnr. \texttt{toString()} implementeras i subklasserna.

\nwenddocs{}\nwbegincode{17}\moddef{Human methods}\endmoddef\nwstartdeflinemarkup\nwenddeflinemarkup
        public String getName() \{ return name; \}
        public String getPnr() \{ return pnr; \}
        @Override public abstract String toString();
\nwendcode{}\nwbegindocs{18}\nwdocspar

\subsection{\texttt{NonBinary}}
\noindent \textbf{Ansvar:} konkret \texttt{Human} för icke\-binär. Klassen är \emph{paket\-synlig}
(ingen \texttt{public}) och \emph{final} så klienten varken kan referera till den
eller ärva utanför paketet.

Filen {\Tt{}human/NonBinary.java\nwendquote} ser ut så här:

\nwenddocs{}\nwbegincode{19}\moddef{human/NonBinary.java}\endmoddef\nwstartdeflinemarkup\nwenddeflinemarkup
package human;

final class NonBinary extends Human \{
        \LA{}NonBinary ctor\RA{}
        \LA{}NonBinary toString\RA{}
\}
\nwendcode{}\nwbegindocs{20}\nwdocspar

\subsubsection{Konstruktor}
Också paket\-synlig (ingen modifierare) — kan inte anropas utanför paketet.

\nwenddocs{}\nwbegincode{21}\moddef{NonBinary ctor}\endmoddef\nwstartdeflinemarkup\nwenddeflinemarkup
        NonBinary(String name, String pnr) \{
                super(name, pnr);
        \}
\nwendcode{}\nwbegindocs{22}\nwdocspar

\subsubsection{\texttt{toString}}
\nwenddocs{}\nwbegincode{23}\moddef{NonBinary toString}\endmoddef\nwstartdeflinemarkup\nwenddeflinemarkup
        @Override
        public String toString() \{
                return "Jag är icke-binär och heter " + name;
        \}
\nwendcode{}\nwbegindocs{24}\nwdocspar

\subsection{\texttt{Woman}}
\noindent \textbf{Ansvar:} konkret \texttt{Human} för kvinna. Paket\-synlig och \texttt{final}.

Filen {\Tt{}human/Woman.java\nwendquote}:

\nwenddocs{}\nwbegincode{25}\moddef{human/Woman.java}\endmoddef\nwstartdeflinemarkup\nwenddeflinemarkup
package human;

final class Woman extends Human \{
        \LA{}Woman ctor\RA{}
        \LA{}Woman toString\RA{}
\}
\nwendcode{}\nwbegindocs{26}\nwdocspar

\subsubsection{Konstruktor}
\nwenddocs{}\nwbegincode{27}\moddef{Woman ctor}\endmoddef\nwstartdeflinemarkup\nwenddeflinemarkup
        Woman(String name, String pnr) \{
                super(name, pnr);
        \}
\nwendcode{}\nwbegindocs{28}\nwdocspar

\subsubsection{\texttt{toString}}
\nwenddocs{}\nwbegincode{29}\moddef{Woman toString}\endmoddef\nwstartdeflinemarkup\nwenddeflinemarkup
        @Override
        public String toString() \{
                return "Jag är kvinna och heter " + name;
        \}
\nwendcode{}\nwbegindocs{30}\nwdocspar

\subsection{\texttt{Man}}
\noindent \textbf{Ansvar:} konkret \texttt{Human} för man. Paket\-synlig och \texttt{final}.

Filen {\Tt{}human/Man.java\nwendquote}:

\nwenddocs{}\nwbegincode{31}\moddef{human/Man.java}\endmoddef\nwstartdeflinemarkup\nwenddeflinemarkup
package human;

final class Man extends Human \{
        \LA{}Man ctor\RA{}
        \LA{}Man toString\RA{}
\}
\nwendcode{}\nwbegindocs{32}\nwdocspar

\subsubsection{Konstruktor}
\nwenddocs{}\nwbegincode{33}\moddef{Man ctor}\endmoddef\nwstartdeflinemarkup\nwenddeflinemarkup
        Man(String name, String pnr) \{
                super(name, pnr);
        \}
\nwendcode{}\nwbegindocs{34}\nwdocspar

\subsubsection{\texttt{toString}}
\nwenddocs{}\nwbegincode{35}\moddef{Man toString}\endmoddef\nwstartdeflinemarkup\nwenddeflinemarkup
        @Override
        public String toString() \{
                return "Jag är man och heter " + name;
        \}
\nwendcode{}\nwbegindocs{36}\nwdocspar

\subsection{\texttt{TestHuman}}
\noindent \textbf{Ansvar:} visa fabriksanvändning och utskrift. Testprogrammet
ligger i default\-paketet och kan endast skapa objekt via \texttt{Human.create}.

Filen {\Tt{}TestHuman.java\nwendquote} ser översiktligt ut så här:

\nwenddocs{}\nwbegincode{37}\moddef{TestHuman.java}\endmoddef\nwstartdeflinemarkup\nwenddeflinemarkup
import human.Human;

public class TestHuman \{
        \LA{}TestHuman main\RA{}
\}
\nwendcode{}\nwbegindocs{38}\nwdocspar

\subsubsection{\texttt{main}}
Vi skapar ett objekt av varje typ via fabriken och skriver ut.

\nwenddocs{}\nwbegincode{39}\moddef{TestHuman main}\endmoddef\nwstartdeflinemarkup\nwenddeflinemarkup
        public static void main(String[] args) \{
                Human billie = Human.create("Billie", "xxxxxx-560x"); // näst sista = 0 -> NonBinary
                Human anna   = Human.create("Anna",   "xxxxxx-642x"); // näst sista = 2 -> Woman
                Human magnus = Human.create("Magnus", "xxxxxx-011x"); // näst sista = 1 -> Man

                System.out.println(billie);
                System.out.println(anna);
                System.out.println(magnus);

                // Följande rader ska INTE kompilera (demonstreras separat, inte i detta program):
                // NonBinary nb = new NonBinary("X", "000000-5600");    // ej synlig utanför paketet
                // Human h = new Human("X", "000000-5600") \{\};          // Human() ej synlig + abstrakt
        \}
\nwendcode{}\nwbegindocs{40}\nwdocspar
\nwenddocs{}\end{document}

