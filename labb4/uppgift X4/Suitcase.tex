\documentclass{article}\usepackage{noweb}\pagestyle{noweb}\noweboptions{}\begin{document}\nwfilename{Suitcase.nw}\nwbegindocs{0}\title{Resväska som \emph{Composite} i Java}% ===> this file was generated automatically by noweave --- better not edit it
\author{Lisa Dahl och Mostafa Shihadeh}

\maketitle
\tableofcontents

\section{Introduktion}
Vi vill implementera uppgift 1 (resväska enligt \emph{Composite}-mönstret)
med litterär programmering.
Vi bygger tre klasser: \texttt{Component} (abstrakt), \texttt{Leaf} (pryl) och
\texttt{Composite} (behållare), samt ett testprogram \texttt{Client}.

I extrauppgiften X4 lägger vi även till två iteratorer för vår Composite-struktur:
en som går \emph{bredden-först} (BFS) och en som går \emph{preorder} (djupet-först).
Då kan vi gå igenom alla noder med en for-each-sats eller med iteratorns 
\texttt{hasNext}/\texttt{next}-metoder utan att bry oss om hur trädet är uppbyggt.

\subsection{Designval}
\begin{itemize}
  \item \texttt{Component} bär gemensamma attribut: \texttt{name} och \texttt{weight} (egen vikt).
        Dessa lagras som \emph{private final}-fält och kan läsas via getters.
  \item \texttt{Leaf} representerar en enskild pryl. \texttt{getWeight()} returnerar bara dess egen vikt.
  \item \texttt{Composite} representerar en behållare med barn. \texttt{getWeight()} summerar
        behållarens egen vikt och alla barns vikter. \texttt{toString()} traverserar rekursivt.
  \item Metoderna \texttt{add} och \texttt{remove} finns bara i \texttt{Composite} (inte i \texttt{Component}).
  \item I X4 låter vi \texttt{Composite} implementera \texttt{Iterable<Component>} och skriver
        två iterator-klasser: \texttt{BreadthFirstIterator} (BFS) och \texttt{PreorderIterator} (DFS).
\end{itemize}

\subsection{Bygga och kompilera}
Vi vill skriva en byggfil för GNU Make. De maskin-genererade reglerna
läggs i \texttt{Suitcase.mk} som tanglas ur denna \texttt{.nw}-fil och sedan
\emph{inkluderas} från toppens \texttt{Makefile}.

Först lägger vi grundmålen och PDF-reglerna:

\nwenddocs{}\nwbegincode{1}\moddef{Suitcase.mk}\endmoddef\nwstartdeflinemarkup\nwenddeflinemarkup
TARGETS= Suitcase.pdf Suitcase.mk
all: classes Suitcase.pdf

Suitcase.pdf:  Suitcase.tex
  pdflatex -interaction=nonstopmode -halt-on-error Suitcase.tex
  pdflatex -interaction=nonstopmode -halt-on-error Suitcase.tex

Suitcase.tex: Suitcase.nw
  noweave -latex Suitcase.nw > Suitcase.tex
\nwendcode{}\nwbegindocs{2}\nwdocspar

Därefter tanglar vi ut varje \texttt{.java}-fil ur den litterära källan
(samt bäddar in radmarkörer så att \texttt{noerr.pl} kan mappa fel till \texttt{.nw}):

\nwenddocs{}\nwbegincode{3}\moddef{Suitcase.mk}\plusendmoddef\nwstartdeflinemarkup\nwenddeflinemarkup
Component.java: Suitcase.nw
  notangle -L'//line %L "%F"%N' -RComponent.java Suitcase.nw > Component.java

Leaf.java: Suitcase.nw
  notangle -L'//line %L "%F"%N' -RLeaf.java Suitcase.nw > Leaf.java

Composite.java: Suitcase.nw
  notangle -L'//line %L "%F"%N' -RComposite.java Suitcase.nw > Composite.java

Client.java: Suitcase.nw
  notangle -L'//line %L "%F"%N' -RClient.java Suitcase.nw > Client.java

BreadthFirstIterator.java: Suitcase.nw
  notangle -L'//line %L "%F"%N' -RBreadthFirstIterator.java Suitcase.nw > BreadthFirstIterator.java

PreorderIterator.java: Suitcase.nw
  notangle -L'//line %L "%F"%N' -RPreorderIterator.java Suitcase.nw > PreorderIterator.java
\nwendcode{}\nwbegindocs{4}\nwdocspar

Nu lägger vi sektionen för att kompilera och köra Java:

\nwenddocs{}\nwbegincode{5}\moddef{Suitcase.mk}\plusendmoddef\nwstartdeflinemarkup\nwenddeflinemarkup
.PHONY: classes run clean-Suitcase
classes: Component.java Leaf.java Composite.java Client.java BreadthFirstIterator.java PreorderIterator.java
  @if [ -x ./noerr.pl ]; then ./noerr.pl javac *.java; else javac *.java; fi

run: classes
  java Client
\nwendcode{}\nwbegindocs{6}\nwdocspar

Slutligen städreglerna:

\nwenddocs{}\nwbegincode{7}\moddef{Suitcase.mk}\plusendmoddef\nwstartdeflinemarkup\nwenddeflinemarkup
clean: clean-Suitcase
clean-Suitcase:
  rm -f Suitcase.tex Suitcase.aux Suitcase.log Suitcase.toc
  rm -f *.class *.java
\nwendcode{}\nwbegindocs{8}\nwdocspar

\section{Kod}
I det här avsnittet definierar vi Javafilerna som tänglas ut från denna \texttt{.nw}-fil.

\subsection{\texttt{Klassen Component}}
\noindent \textbf{Ansvar:} bas-klass med namn och egenvikt samt abstrakta metoder.

Filen {\Tt{}Component.java\nwendquote} innehåller definitionen av klassen \texttt{Component}.
Översiktligt ser den ut så här:
\nwenddocs{}\nwbegincode{9}\moddef{Component.java}\endmoddef\nwstartdeflinemarkup\nwenddeflinemarkup
public abstract class Component \{
  \LA{}Component attributes\RA{}
  \LA{}Component constructor\RA{}
  \LA{}Component methods\RA{}
\}
\nwendcode{}\nwbegindocs{10}\nwdocspar

\subsubsection{Attribut}
Vi vill att \texttt{Component} ska ha följande attribut:
\begin{itemize}
  \item \texttt{name}: namn på komponenten (sträng)
  \item \texttt{weight}: egen vikt i kg (flyttal)
\end{itemize}

Dessa attribut ska inte kunna ändras efter att objektet skapats. 
Vi markerar dem därför som \texttt{private final} och ger enkla getters.

\nwenddocs{}\nwbegincode{11}\moddef{Component attributes}\endmoddef\nwstartdeflinemarkup\nwenddeflinemarkup
  private final String name;
  private final double weight;
\nwendcode{}\nwbegindocs{12}\nwdocspar

\subsubsection{Konstruktor}
Konstruktorn tar emot namn och vikt och lagrar dem.

\nwenddocs{}\nwbegincode{13}\moddef{Component constructor}\endmoddef\nwstartdeflinemarkup\nwenddeflinemarkup
  Component(String name, double weight) \{
    this.name = name;
    this.weight = weight;
  \}
\nwendcode{}\nwbegindocs{14}\nwdocspar

\subsubsection{Metoder}
I \texttt{Component} deklarerar vi getters för namnet och egenvikten, samt de
abstrakta metoderna \texttt{getWeight} och \texttt{toString} som alla 
subklasser måste implementera:
\begin{itemize}
  \item \texttt{getName}: returnerar komponentens namn.
  \item \texttt{getOwnWeight}: returnerar komponentens egen vikt (utan barn).
  \item \texttt{getWeight}: ska returnera totalvikten (inklusive barns vikt).
  \item \texttt{toString}: ska returnera en strängrepresentation av komponenten.
\end{itemize}

\nwenddocs{}\nwbegincode{15}\moddef{Component methods}\endmoddef\nwstartdeflinemarkup\nwenddeflinemarkup
  public String getName() \{
    return name;
  \}

  public double getOwnWeight() \{
    return weight;
  \}

  public abstract double getWeight();
  public abstract String toString();
\nwendcode{}\nwbegindocs{16}\nwdocspar

\subsection{\texttt{Leaf}}
\noindent \textbf{Ansvar:} en enskild pryl att packa.

Filen {\Tt{}Leaf.java\nwendquote} innehåller definitionen av klassen \texttt{Leaf}.
Översiktligt ser den ut så här:

\nwenddocs{}\nwbegincode{17}\moddef{Leaf.java}\endmoddef\nwstartdeflinemarkup\nwenddeflinemarkup
public class Leaf extends Component \{
  \LA{}Leaf constructor\RA{}
  \LA{}Leaf methods\RA{}
\}
\nwendcode{}\nwbegindocs{18}\nwdocspar

\subsubsection{Konstruktor}
Konstruktorn tar namn och egen vikt och skickar vidare till basklassen \texttt{Component}.

\nwenddocs{}\nwbegincode{19}\moddef{Leaf constructor}\endmoddef\nwstartdeflinemarkup\nwenddeflinemarkup
  public Leaf(String name, double weight) \{
    super(name, weight);
  \}
\nwendcode{}\nwbegindocs{20}\nwdocspar

\subsubsection{Metoder}
Totalvikten för ett löv är samma som dess egen vikt. \texttt{toString} beskriver prylen.

\nwenddocs{}\nwbegincode{21}\moddef{Leaf methods}\endmoddef\nwstartdeflinemarkup\nwenddeflinemarkup
  @Override
  public double getWeight() \{
    return getOwnWeight();
  \}

  @Override
  public String toString() \{
    return getName() + " (" + getOwnWeight() + ")";
  \}
\nwendcode{}\nwbegindocs{22}\nwdocspar

\subsection{\texttt{Composite}}
\noindent \textbf{Ansvar:} en behållare som kan innehålla andra \texttt{Component}.
Egen vikt \emph{plus} alla barns vikter utgör totalvikten. I X4 gör vi den också
itererbar så att vi kan gå igenom trädet med iteratorer.

Filen {\Tt{}Composite.java\nwendquote} innehåller definitionen av klassen \texttt{Composite}.
Översiktligt ser den ut så här:

\nwenddocs{}\nwbegincode{23}\moddef{Composite.java}\endmoddef\nwstartdeflinemarkup\nwenddeflinemarkup
import java.util.*;

public class Composite extends Component implements Iterable<Component> \{
  \LA{}Composite fields\RA{}
  \LA{}Composite constructor\RA{}
  \LA{}Composite methods\RA{}
  \LA{}Composite iterator methods\RA{}
\}
\nwendcode{}\nwbegindocs{24}\nwdocspar

\subsubsection{Fält}
Vi lagrar barn i en muterbar lista av \texttt{Component}.

\nwenddocs{}\nwbegincode{25}\moddef{Composite fields}\endmoddef\nwstartdeflinemarkup\nwenddeflinemarkup
  private final List<Component> children = new ArrayList<>();
\nwendcode{}\nwbegindocs{26}\nwdocspar

\subsubsection{Konstruktor}
Konstruktorn tar behållarens namn och egen vikt.

\nwenddocs{}\nwbegincode{27}\moddef{Composite constructor}\endmoddef\nwstartdeflinemarkup\nwenddeflinemarkup
  public Composite(String name, double weight) \{
    super(name, weight);
  \}
\nwendcode{}\nwbegindocs{28}\nwdocspar

\subsubsection{Metoder}
\paragraph{Barnhantering (\texttt{add}/\texttt{remove}/\texttt{getChild}/\texttt{getChildren})}
Vi kan lägga till och ta bort barn, och hämta ett barn via index eller läsa alla barn.

\nwenddocs{}\nwbegincode{29}\moddef{Composite methods}\endmoddef\nwstartdeflinemarkup\nwenddeflinemarkup
  public void add(Component component) \{
    children.add(component);
  \}

  public void remove(Component component) \{
    children.remove(component);
  \}

  public Component getChild(int index) \{
    return children.get(index);
  \}

  public List<Component> getChildren() \{
    return Collections.unmodifiableList(children);
  \}
\nwendcode{}\nwbegindocs{30}\nwdocspar

\paragraph{Totalvikt (\texttt{getWeight})}
Totalvikt = behållarens egen vikt + summan av alla barns totalvikter (rekursivt).
Varje anrop på ett \texttt{Composite}-objekt går igenom dess barn och anropar
deras \texttt{getWeight()} enligt mönstret.

\nwenddocs{}\nwbegincode{31}\moddef{Composite methods}\plusendmoddef\nwstartdeflinemarkup\nwenddeflinemarkup
  @Override
  public double getWeight() \{
    double totalWeight = getOwnWeight();
    for (Component child : children) \{
      totalWeight += child.getWeight();
    \}
    return totalWeight;
  \}
\nwendcode{}\nwbegindocs{32}\nwdocspar

\paragraph{\texttt{toString}}
Vi bygger en rekursiv beskrivning där behållarens namn och egen vikt visas,
och alla barn listas inom hakparenteser. Även här anropar vi barnens 
\texttt{toString()} enligt mönstret.

\nwenddocs{}\nwbegincode{33}\moddef{Composite methods}\plusendmoddef\nwstartdeflinemarkup\nwenddeflinemarkup
  @Override
  public String toString() \{
    StringBuilder sb = new StringBuilder();
    sb.append(getName()).append(" (").append(getOwnWeight()).append(") [");
    for (int i = 0; i < children.size(); i++) \{
      sb.append(children.get(i).toString());
      if (i < children.size() - 1) \{
        sb.append(", ");
      \}
    \}
    sb.append("]");

    return sb.toString();
  \}
\nwendcode{}\nwbegindocs{34}\nwdocspar

\paragraph{Iterator-metoder (X4)}
\texttt{Composite} implementerar \texttt{Iterable<Component>}. Default\-iteratorn
är bredden-först. Vi lägger också till metoder för att uttryckligen få BFS respektive preorder.

\nwenddocs{}\nwbegincode{35}\moddef{Composite iterator methods}\endmoddef\nwstartdeflinemarkup\nwenddeflinemarkup
  @Override
  public Iterator<Component> iterator() \{
    // default: bredden-först
    return new BreadthFirstIterator(this);
  \}

  public Iterator<Component> breadthFirstIterator() \{
    return new BreadthFirstIterator(this);
  \}

  public Iterator<Component> preorderIterator() \{
    return new PreorderIterator(this);
  \}
\nwendcode{}\nwbegindocs{36}\nwdocspar

\subsection{\texttt{BreadthFirstIterator} (X4)}
\noindent \textbf{Ansvar:} gå igenom trädet i bredden-först-ordning:
rot, alla barn, alla barnbarn, osv.

Filen {\Tt{}BreadthFirstIterator.java\nwendquote} innehåller iteratorn:

\nwenddocs{}\nwbegincode{37}\moddef{BreadthFirstIterator.java}\endmoddef\nwstartdeflinemarkup\nwenddeflinemarkup
import java.util.ArrayDeque;
import java.util.Deque;
import java.util.Iterator;
import java.util.NoSuchElementException;

public class BreadthFirstIterator implements Iterator<Component> \{
  private final Deque<Component> queue = new ArrayDeque<>();

  // starta med roten i kön
  public BreadthFirstIterator(Component root) \{
    queue.add(root);
  \}

  @Override
  public boolean hasNext() \{
    return !queue.isEmpty();
  \}

  @Override
  public Component next() \{
    if (!hasNext()) \{
      throw new NoSuchElementException();
    \}
    Component current = queue.removeFirst();
    if (current instanceof Composite) \{
      for (Component child : ((Composite) current).getChildren()) \{
        queue.addLast(child);
      \}
    \}
    return current;
  \}

  @Override
  public void remove() \{
    // används inte i den här labben
  \}
\}
\nwendcode{}\nwbegindocs{38}\nwdocspar

\subsection{\texttt{PreorderIterator} (X4)}
\noindent \textbf{Ansvar:} gå igenom trädet i preorder (djupet-först):
besök noden, sedan rekursivt dess barn från vänster till höger.

Filen {\Tt{}PreorderIterator.java\nwendquote} innehåller iteratorn:

\nwenddocs{}\nwbegincode{39}\moddef{PreorderIterator.java}\endmoddef\nwstartdeflinemarkup\nwenddeflinemarkup
import java.util.ArrayDeque;
import java.util.Deque;
import java.util.Iterator;
import java.util.List;
import java.util.NoSuchElementException;

public class PreorderIterator implements Iterator<Component> \{
  private final Deque<Component> stack = new ArrayDeque<>();

  // börja med roten överst på stacken
  public PreorderIterator(Component root) \{
    stack.push(root);
  \}

  @Override
  public boolean hasNext() \{
    return !stack.isEmpty();
  \}

  @Override
  public Component next() \{
    if (!hasNext()) \{
      throw new NoSuchElementException();
    \}
    Component current = stack.pop();
    if (current instanceof Composite) \{
      List<Component> children = ((Composite) current).getChildren();
      // lägg barnen i omvänd ordning så att första barnet kommer överst
      for (int i = children.size() - 1; i >= 0; i--) \{
        stack.push(children.get(i));
      \}
    \}
    return current;
  \}

  @Override
  public void remove() \{
    // används inte i den här labben
  \}
\}
\nwendcode{}\nwbegindocs{40}\nwdocspar


\subsection{\texttt{Client}}
\noindent \textbf{Ansvar:} bygga en resväska med minst tre nivåer och minst tio prylar,
skriva ut totalvikt och innehåll, ta bort några objekt och skriva ut igen.
I X4 visar vi också hur vi går igenom trädet med både for-each och explicit iterator.

Filen {\Tt{}Client.java\nwendquote} innehåller testprogrammet. Översiktligt ser det ut så här:

\nwenddocs{}\nwbegincode{41}\moddef{Client.java}\endmoddef\nwstartdeflinemarkup\nwenddeflinemarkup
public class Client \{
  public static void main(String[] args) \{
    \LA{}Client build-structure\RA{}
    \LA{}Client print\RA{}
    \LA{}Client iterate-bfs\RA{}
    \LA{}Client iterate-preorder\RA{}
  \}
\}
\nwendcode{}\nwbegindocs{42}\nwdocspar

\subsubsection{Bygg upp strukturen (\texttt{suitcase})}
Vi skapar roten (resväskan), underbehållare och löv, som är kläder eller accessoarer.

\nwenddocs{}\nwbegincode{43}\moddef{Client build-structure}\endmoddef\nwstartdeflinemarkup\nwenddeflinemarkup
    // Roten: själva resväskan
    Composite suitcase = new Composite("Resväska", 2.3);

    // Necessär (behållare) med egen vikt 0.12 kg och innehåll
    Composite necessar = new Composite("Necessär", 0.12);
    necessar.add(new Leaf("Tvål",        0.09));
    necessar.add(new Leaf("Schampo",     0.22));
    necessar.add(new Leaf("Tandborste",  0.03));
    necessar.add(new Leaf("Tandkräm",    0.11));

    // Påse i necessären (tredje nivån)
    Composite pase = new Composite("Påse", 0.01);
    pase.add(new Leaf("Hårspännen (10 st)", 0.02));
    necessar.add(pase);

    // Mindre väska (nivå två)
    Composite techbag = new Composite("Tech-väska", 0.20);
    techbag.add(new Leaf("Laddare", 0.15));

    // Packa allt i resväskan
    suitcase.add(new Leaf("T-shirt vit",   0.18));
    suitcase.add(new Leaf("T-shirt svart", 0.19));
    suitcase.add(new Leaf("Jeans",         0.75));
    suitcase.add(new Leaf("Chinos",        0.55));
    suitcase.add(new Leaf("Pocketbok",     0.28));
    suitcase.add(necessar);
    suitcase.add(techbag);
\nwendcode{}\nwbegindocs{44}\nwdocspar

\subsubsection{Skriv ut totalvikt och innehåll (före borttagning)}
\nwenddocs{}\nwbegincode{45}\moddef{Client print}\endmoddef\nwstartdeflinemarkup\nwenddeflinemarkup
    System.out.printf("Totalvikt före borttagning: %.2f kg%n", suitcase.getWeight());
    System.out.println("Innehåll före borttagning:");
    System.out.println(suitcase.toString());
\nwendcode{}\nwbegindocs{46}\nwdocspar


\subsubsection{X4: Traversal med iteratorer}
För att kontrollera ordningen skriver vi bara ut nodernas namn med \texttt{getName()}.

\nwenddocs{}\nwbegincode{47}\moddef{Client iterate-bfs}\endmoddef\nwstartdeflinemarkup\nwenddeflinemarkup
    System.out.println("Bredden-först (for-each, default iterator):");
    for (Component c : suitcase) \{
      System.out.print(c.getName() + " ");
    \}
    System.out.println();
\nwendcode{}\nwbegindocs{48}\nwdocspar

\nwenddocs{}\nwbegincode{49}\moddef{Client iterate-preorder}\endmoddef\nwstartdeflinemarkup\nwenddeflinemarkup
    System.out.println("Preorder (djupet-först, explicit iterator):");
    java.util.Iterator<Component> it = new PreorderIterator(suitcase);
    while (it.hasNext()) \{
      System.out.print(it.next().getName() + " ");
    \}
    System.out.println();
\nwendcode{}\nwbegindocs{50}\nwdocspar
\nwenddocs{}\end{document}

